\chapter{Planteamiento del Problema}
\section{Descripción de la Realidad Problemática}

En el Perú, la inseguridad ciudadana ha alcanzado niveles alarmantes, particularmente en zonas urbanas como Lima y el Callao, que albergan más de un tercio de la población del país. La situación se ha agravado a lo largo de los años, impulsada tanto por un aumento en la frecuencia de delitos como por la percepción generalizada de inseguridad en la población. De acuerdo con el Instituto Nacional de Estadística e Informática (INEI), el 88.4\%  de la población de Lima Metropolitana y Callao expresó temor de ser víctima de un delito en los próximos doce meses en el primer semestre de 2024. Esta percepción de inseguridad refleja una tendencia en aumento, influida por la constante amenaza de robos, asaltos, y otros crímenes que ocurren con regularidad en la ciudad.

La victimización en Lima Metropolitana ha sido una constante en las encuestas de seguridad ciudadana. Según las estadísticas del primer semestre de 2024, un promedio de 13 de cada 100 habitantes en las principales ciudades del país ha sido víctima de algún tipo de robo en la vía pública. Los delitos más comunes incluyen robos de dinero, celulares y carteras, con una prevalencia particularmente alta en espacios públicos y en los sistemas de transporte urbano. Estos actos delictivos no solo afectan a las víctimas de manera directa, sino que generan un ambiente de inseguridad y miedo constante en la población, que ve afectada su calidad de vida al sentirse vulnerable en su entorno.

Uno de los factores que agrava esta situación es la violencia con la que se llevan a cabo muchos de estos delitos. En Lima Metropolitana, un porcentaje significativo de los robos y asaltos se cometen bajo amenazas de violencia física, con delincuentes que portan armas de fuego o cuchillos. Esta realidad incrementa el riesgo de que las víctimas sufran lesiones graves e incluso mortales. Los datos oficiales indican que un alto porcentaje de los robos en la ciudad involucra algún tipo de arma, lo que eleva la peligrosidad de estos delitos y aumenta el impacto psicológico en la población. Según los registros de la Policía Nacional, los robos violentos y con armas representan casi el 60\%  de todos los delitos registrados en las zonas urbanas de Lima, un porcentaje alarmante que subraya la vulnerabilidad de la población y la urgencia de medidas preventivas más efectivas.

A pesar de los esfuerzos del gobierno y de las autoridades locales para implementar cámaras de vigilancia y aumentar el patrullaje policial, los sistemas actuales no han logrado reducir de manera efectiva los índices de criminalidad. Las cámaras de seguridad instaladas en puntos estratégicos de la ciudad están mayormente limitadas a registrar los eventos en tiempo real sin capacidad predictiva, lo que significa que su utilidad se restringe principalmente a la recolección de evidencia después de que el crimen ha ocurrido. Además, estos sistemas dependen de la supervisión humana, lo que presenta problemas de limitación de recursos y de capacidad para monitorear continuamente grandes áreas urbanas con alta actividad. En muchos casos, la intervención policial solo ocurre una vez que el delito ya ha sido cometido, lo que reduce las posibilidades de prevenir daños a la propiedad y a las personas.
La percepción de inseguridad en Lima se ve agravada por la falta de un sistema de vigilancia que permita actuar de manera proactiva ante la actividad delictiva. La adaptación de los delincuentes a las tecnologías de seguridad actuales ha aumentado la dificultad para prevenir y reducir la criminalidad. Muchos delincuentes emplean ahora estrategias para evitar la identificación, como el uso de gorras, capuchas y otros elementos que ocultan sus rasgos faciales y corporales. Este tipo de tácticas se ha convertido en una práctica común, lo que complica la identificación de los sospechosos y limita la efectividad de los sistemas de videovigilancia tradicionales. Esta situación revela una necesidad urgente de implementar tecnologías avanzadas que permitan detectar y analizar patrones de comportamiento en tiempo real para emitir alertas preventivas antes de que se cometan delitos.

En este contexto, el desarrollo de sistemas de vigilancia basados en inteligencia artificial (IA) y visión por computadora emerge como una solución potencialmente eficaz para abordar las limitaciones de los sistemas actuales. Estos sistemas tienen la capacidad de analizar grandes volúmenes de datos visuales y detectar patrones de comportamiento que sugieran una amenaza inminente, como movimientos repetitivos en áreas sensibles, gestos o posturas sospechosas, y la utilización de atuendos que ocultan la identidad de los delincuentes. A diferencia de los métodos tradicionales de vigilancia, que dependen de la supervisión humana, un sistema de vigilancia inteligente basado en IA podría identificar indicadores de riesgo y emitir alertas en tiempo real, facilitando la intervención inmediata de las fuerzas de seguridad.
En el caso de Lima y otras ciudades urbanas del Perú, la implementación de un sistema de visión por computadora que permita anticipar comportamientos delictivos representa una medida innovadora que podría reducir significativamente los índices de criminalidad. El uso de algoritmos de detección basados en IA permitiría a los sistemas de vigilancia identificar con precisión los comportamientos sospechosos, minimizando las alarmas falsas y optimizando la intervención de las autoridades. Esta tecnología también podría ser de gran ayuda para reducir la sobrecarga de trabajo de las fuerzas policiales, quienes actualmente se ven obligadas a supervisar grandes volúmenes de imágenes y a responder de manera reactiva a los eventos criminales.

La adopción de tecnologías de IA en el ámbito de la seguridad pública no solo contribuiría a reducir la criminalidad, sino que también podría ayudar a restaurar la confianza de la ciudadanía en las instituciones de seguridad. Un sistema que permita detectar y analizar comportamientos delictivos antes de que estos se materialicen, mediante la emisión de alertas tempranas, brindaría una capa adicional de seguridad a la población. Esto podría ayudar a mitigar el miedo constante que afecta la vida diaria de los ciudadanos, quienes actualmente limitan sus actividades y modifican sus rutinas debido al temor de ser víctimas de un delito.



\section{Formulación del Problema}

\subsection{Problema General}
PG: \newcommand{\ProblemaGeneral}{
¿Cómo puede un sistema de visión por computadora e inteligencia artificial prever y analizar comportamientos delictivos para emitir alertas tempranas y mejorar la seguridad en zonas urbanas de alto riesgo?
}
\ProblemaGeneral
\subsection{Problemas Específicos}
\newcommand{\Pbone}{
¿Por qué no existen suficientes herramientas tecnológicas que permitan detectar comportamientos delictivos en tiempo real?
}
\newcommand{\Pbtwo}{
¿Qué dificultades existen en el uso de simulaciones predictivas para predecir comportamientos delictivos?
}
\newcommand{\Pbthree}{
¿Cómo se pueden mejorar los sistemas de videovigilancia mediante algoritmos de IA?
}


\begin{itemize}
	\item PE1: {\Pbone}
	\item PE2: {\Pbtwo}
	\item PE3: {\Pbthree}
\end{itemize}

\section{Objetivos de la Investigación}
A continuación, se presentan el objetivo general y los objetivos específicos.
\subsection{Objetivo General}
OG: \newcommand{\ObjetivoGeneral}{
Desarrollar un sistema basado en IA y visión por computadora que permita detectar y predecir comportamientos delictivos en tiempo real, con el objetivo de emitir alertas tempranas para mejorar la seguridad en entornos urbanos.
}
\ObjetivoGeneral
\subsection{Objetivos Específicos}
\newcommand{\Objone}{
Diseñar algoritmos de visión por computadora para la identificación de posturas y elementos de disfraz como gorras o capuchas.
}
\newcommand{\Objtwo}{
Implementar un modelo predictivo que analice patrones de comportamiento sospechoso y determine la probabilidad de que ocurra un delito.
}
\newcommand{\Objthree}{
Validar el sistema mediante pruebas en escenarios simulados y analizar su precisión en la emisión de alertas tempranas.
}

\begin{itemize}
	\item OE1: {\Objone}
	\item OE2: {\Objtwo}
	\item OE3: {\Objthree}
\end{itemize}

\section{Justificación de la Investigación}

\subsection{Teórica}
La investigación sobre el uso de visión por computadora e inteligencia artificial para la detección de comportamientos sospechosos y la emisión de alertas tempranas en tiempo real representa una contribución significativa al campo de la seguridad pública y las tecnologías de vigilancia preventiva. La capacidad de analizar patrones de comportamiento a través de algoritmos avanzados no solo amplía los alcances de la tecnología actual, sino que también permite comprender y adaptar modelos predictivos para contextos urbanos de alta densidad, como Lima, donde las amenazas a la seguridad ciudadana son constantes. Al integrar enfoques de IA en sistemas de vigilancia, esta investigación aborda un vacío en el conocimiento sobre la aplicación de técnicas de visión por computadora para la predicción y prevención del crimen en tiempo real.

\subsection{Práctica}
El impacto práctico de implementar un sistema de vigilancia inteligente que detecte comportamientos sospechosos en tiempo real es significativo para el contexto urbano de Lima y otras ciudades con problemas similares de criminalidad. Actualmente, los métodos tradicionales de vigilancia resultan insuficientes y reactivos, ya que actúan después de que los delitos han ocurrido. Un sistema basado en IA y visión por computadora tiene el potencial de prevenir actos delictivos al anticiparse a ellos, lo cual no solo reduce los índices de criminalidad, sino que también disminuye la sensación de inseguridad en la población. La adopción de este tipo de tecnología puede optimizar los recursos de las fuerzas de seguridad, facilitando una respuesta rápida y eficaz ante incidentes potenciales, lo que resulta especialmente útil en zonas con limitaciones en el número de personal de vigilancia y en áreas de alta incidencia delictiva.

\subsection{Metodológica}
Desde una perspectiva metodológica, esta investigación utiliza enfoques avanzados de aprendizaje automático y visión por computadora para desarrollar un sistema capaz de procesar imágenes y videos en tiempo real, diferenciando entre comportamientos normales y sospechosos. La metodología incluye la aplicación de modelos de redes neuronales convolucionales (CNN), algoritmos de detección de patrones y aprendizaje profundo para maximizar la precisión en la detección de amenazas potenciales. Al emplear un enfoque de simulación y pruebas en escenarios controlados, la investigación no solo validará la eficacia del sistema propuesto, sino que también contribuirá a establecer protocolos y estándares metodológicos para la implementación de sistemas de vigilancia inteligente en entornos urbanos complejos.


\section{Delimitación del Estudio}
A continuación, se presentará la delimitación espacial, temporal y conceptual.

\subsection{Espacial}
La investigación se centra en el contexto urbano de Lima, Perú, específicamente en zonas de alto riesgo delictivo caracterizadas por una alta densidad poblacional y una elevada incidencia de robos, asaltos y otros delitos violentos. Este enfoque permitirá evaluar el rendimiento del sistema en escenarios reales de vigilancia urbana, donde los retos de seguridad y las limitaciones de los sistemas tradicionales son especialmente evidentes. Aunque el estudio se limita inicialmente a Lima, los hallazgos y las metodologías desarrolladas podrían adaptarse a otras ciudades de características similares.

\subsection{Temporal}
El proyecto se desarrollará entre 2024 y 2026, un periodo que incluye las fases de diseño, implementación y validación del sistema propuesto. Este marco temporal permite la recolección de datos suficientes para evaluar el rendimiento del sistema de manera exhaustiva, así como el ajuste de los algoritmos en función de los resultados obtenidos durante la fase de pruebas y simulación en entornos controlados.

\subsection{Conceptual}
Los conceptos clave de la investigación incluyen “comportamiento sospechoso”, “detección en tiempo real”, “visión por computadora” e “inteligencia artificial aplicada a la vigilancia”. El concepto de “comportamiento sospechoso” se refiere a patrones de conducta que, basados en estudios previos y análisis de datos históricos, suelen anteceder la comisión de delitos. La “detección en tiempo real” implica que el sistema debe procesar y analizar datos en el momento en que se generan, permitiendo una intervención inmediata. La “visión por computadora” y la “inteligencia artificial aplicada a la vigilancia” representan el conjunto de técnicas y algoritmos utilizados para reconocer patrones de comportamiento y emitir alertas de manera autónoma.
