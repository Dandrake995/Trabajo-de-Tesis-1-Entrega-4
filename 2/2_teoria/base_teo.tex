\subsection{Visión por Computadora}
La visión por computadora es un campo de la inteligencia artificial que se centra en permitir que los sistemas informáticos interpreten y comprendan el contenido visual de imágenes y videos, replicando en cierto sentido el proceso perceptual humano. Este campo se ha expandido considerablemente gracias a los avances en aprendizaje profundo, específicamente en el análisis y procesamiento de imágenes. La visión por computadora es crucial en aplicaciones de seguridad y vigilancia, ya que permite detectar patrones de comportamiento en tiempo real. Según Gonzalez y Woods (2018), los sistemas de visión artificial son especialmente útiles en la detección de actividades sospechosas al combinar técnicas de segmentación, reconocimiento de objetos y análisis de movimiento. Esto hace que la visión por computadora sea una herramienta esencial para el análisis de videos en la vigilancia urbana.

\subsection{Redes Neuronales Convolucionales (CNN)}
Las Redes Neuronales Convolucionales (CNN) son un tipo de arquitectura de red neuronal diseñada específicamente para el procesamiento de datos con una estructura en cuadrícula, como imágenes. Las CNN utilizan capas de convolución que extraen características jerárquicas, comenzando con bordes y texturas básicas y avanzando hacia la identificación de patrones complejos. Esto las hace ideales para aplicaciones de detección de comportamiento en tiempo real. Según Krizhevsky et al. (2012), el modelo AlexNet revolucionó el campo de la visión por computadora al mostrar la capacidad de las CNN para clasificar imágenes con una precisión nunca antes lograda. Desde entonces, las CNN han sido ampliamente adoptadas para la vigilancia, permitiendo la identificación de comportamientos sospechosos mediante el análisis de patrones visuales en videos de vigilancia.

\subsection{Redes LSTM y GRU}
Las redes Long Short-Term Memory (LSTM) y Gated Recurrent Unit (GRU) son tipos de redes neuronales recurrentes diseñadas para manejar secuencias de datos y capturar dependencias temporales a largo plazo. En la vigilancia de seguridad, estas redes son especialmente útiles para analizar patrones de movimiento a lo largo del tiempo, permitiendo detectar anomalías en secuencias de video. Las LSTM, propuestas por Hochreiter y Schmidhuber (1997), introdujeron una estructura de “puertas” que permite a la red mantener información relevante durante largos períodos, lo que resuelve el problema del gradiente en las redes recurrentes tradicionales. Las GRU, por otro lado, son una variante más eficiente y simplificada de las LSTM que ofrece un rendimiento similar con menor consumo computacional, según Cho et al. (2014).


\subsection{YOLO (You Only Look Once)}
You Only Look Once (YOLO) es una técnica de detección de objetos en tiempo real que permite la identificación de múltiples objetos en una sola pasada de la imagen, logrando una gran rapidez en el procesamiento. Redmon et al. (2016) introdujeron YOLO como un modelo único que considera la detección de objetos como un problema de regresión, proporcionando tanto la ubicación como la clasificación de los objetos en un solo paso. Esta técnica es altamente eficaz en sistemas de videovigilancia que requieren una respuesta inmediata, ya que reduce significativamente la latencia y permite una supervisión continua en áreas de alta seguridad.

\subsection{Computación en el Borde}
La computación en el borde es una arquitectura de red en la que el procesamiento de los datos se realiza cerca de la fuente de generación de los mismos, como en cámaras de seguridad o dispositivos de vigilancia, en lugar de depender de servidores remotos en la nube. Esto permite una reducción considerable en la latencia y en el ancho de banda necesario para enviar los datos. Según Shi et al. (2016), la computación en el borde mejora la eficiencia en aplicaciones de vigilancia y facilita una respuesta rápida ante eventos sospechosos. Además, permite que los sistemas de seguridad funcionen incluso con conexiones intermitentes, ya que no dependen de la nube para el procesamiento de los datos.

\subsection{Aprendizaje de Instancias Múltiples (MIL)}
El aprendizaje de instancias múltiples (MIL) es un enfoque en el que los datos se organizan en bolsas de instancias y, a diferencia de los métodos tradicionales de aprendizaje supervisado, las etiquetas se asignan a las bolsas y no a las instancias individuales. Dietterich et al. (1997) introdujeron el MIL para abordar problemas en los que las anotaciones a nivel de instancia no están disponibles. Este método es particularmente útil en la vigilancia, ya que los eventos anómalos en videos pueden no estar detalladamente etiquetados. En lugar de requerir etiquetas precisas para cada cuadro, el MIL permite que los sistemas identifiquen eventos sospechosos al analizar secuencias completas de video, lo cual es fundamental para el análisis de actividades en tiempo real en vigilancia urbana.