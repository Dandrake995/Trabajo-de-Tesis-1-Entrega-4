%La línea de abajo es para quitar encabezado
%\thispagestyle{plain}

\chapter*{Introducción}
\markboth{Introducción}{Introducción}
\addcontentsline{toc}{chapter}{Introducción}
La inseguridad ciudadana en entornos urbanos es uno de los problemas más críticos de nuestra sociedad actual. En ciudades como Lima, Perú, los altos índices de criminalidad y la percepción de inseguridad afectan la calidad de vida de los habitantes. Según estudios recientes, la instalación de cámaras de videovigilancia ha crecido significativamente en espacios públicos, pero estos sistemas siguen siendo predominantemente reactivos, limitados al registro de eventos pasados, y no ofrecen capacidades predictivas Esto resulta en un uso ineficiente de los recursos, ya que se estima que el monitoreo manual de cámaras tiene una tasa de detección de incidentes del 20\% al 30\%, principalmente debido a la fatiga y limitaciones humanas.

Ante este panorama, la inteligencia artificial (IA) y la visión por computadora emergen como herramientas disruptivas para transformar la seguridad urbana. En este contexto, tecnologías como las redes neuronales convolucionales (CNN) y los modelos avanzados de detección en tiempo real, como YOLO y LSTM, permiten la identificación proactiva de comportamientos sospechosos en datos de video. Por ejemplo, se ha demostrado que algoritmos como ConvLSTM pueden alcanzar tasas de precisión superiores al 91\% en la identificación de actividades anómalas, superando significativamente los métodos tradicionales.Además, estos sistemas pueden procesar volúmenes masivos de datos en tiempo real, lo que los hace escalables y adaptables a diferentes escenarios urbanos.

El presente proyecto propone el diseño e implementación de un sistema predictivo de visión por computadora basado en IA, enfocado en la detección temprana de comportamientos delictivos en Lima. Este sistema integrará tecnologías avanzadas como CNN y YOLO para analizar imágenes estáticas y modelos LSTM para reconocer patrones temporales, permitiendo la identificación de comportamientos sospechosos antes de que ocurran incidentes. Como respaldo, estudios previos han demostrado que sistemas similares son efectivos no solo en la detección de comportamientos individuales, sino también en el manejo de multitudes y la identificación de objetos abandonados, logrando reducir incidentes en un 40\% en entornos controlados.

Este trabajo no solo aborda un problema práctico con impacto directo en la seguridad pública, sino que también contribuye al desarrollo teórico de modelos predictivos de IA. Al combinar la innovación tecnológica con un enfoque metodológico riguroso, este proyecto busca establecer un modelo que pueda replicarse en otras ciudades con problemas similares. De esta manera, el sistema propuesto no solo mejorará la seguridad urbana en Lima, sino que también servirá como un modelo para futuras aplicaciones en el ámbito global.
